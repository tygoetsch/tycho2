\documentclass[12pt,letterpaper]{article}
\usepackage[utf8]{inputenc}
\usepackage{amsmath}
\usepackage{amsfonts}
\usepackage{amssymb}
\usepackage{graphicx}
%\usepackage[left=2cm,right=2cm,top=2cm,bottom=2cm]{geometry}
\author{Kris Garrett}
\title{Sweep Schedule}
\begin{document}
\maketitle

\section{Priorities}
Two sets of priorities are created: (i) intra-angle and (ii) inter-angle.
The prioritization heuristics come from the original Tycho code described in ``An Algorithm for Parallel S$_n$ Sweeps on Unstructured Meshes'' by Shawn Pautz.

First, intra-angle priorities are found.
This means we split all the cell-angle pairs into sets of cells where each set corresponds to a distinct angle.
Then each set has priorities applied to every cell.
Second, a global priority is set for each cell-angle pair as a function of the intra-angle priority for each cell and the angle index.



\subsection{Intra-angle Priorities}
There are 5 priority types here.
All use the B-level in the graph for the given angle.
The B-level is defined as the longest path to get to a graph leaf.
A leaf has B-level 1.

\paragraph{Random}
Assign a random priority to each cell.

\paragraph{B-Levels}
Cell priority equals the b-level.

\paragraph{BFDS - Breadth First Descendant Seeking}
Priority equals the highest b-level of any descendant on another process.
If the cell has no descendants on another process, the priority is zero.

\paragraph{DFDS - Depth First Descendant Seeking}
If a cell has no descendants on another process, the priority is zero.
If the cell has a child on another process, the priority is the highest b-level of the children on the other process \textit{plus} a constant greater than or equal to the number of levels of the graph.
Otherwise, the cell has priority one less than the max of the priority of its children.

\paragraph{DFHDS - Depth First Highest Descendant Seeking}
If a cell has no descendants on another process, the priority is zero.
If the cell has a child on another process, the priority is the highest b-level of the children on the other process \textit{times} a constant greater than or equal to the number of levels of the graph.
Otherwise, the cell has priority one less than the max of the priority of its children.


\subsection{Inter-angle Priorities}
Given the intra-angle priorites, there are 3 types of inter-angle priorities.


\paragraph{Interleaved}
This makes the global priority equal to the intra-angle priority.
This will interleave the ordering of cell-angle pairs in angle.

\paragraph{Global}
This makes the global priority equal to the intra-angle priority plus a constant large enough to ensure each angle of lower index are prioritized before angles of higher index.

\paragraph{Local}
Highest b-level is prioritized first.



\section{SweepSchedule}
A sweep schedule is made which given priorities, orders the calculations for the sweep.
The idea is that a \textit{step} is something that can be solved in each process without communication between processes.
These steps are to be done on each process together.
A maximum of \textit{MaxCellsPerStep} cell-angle pairs may be computed for each step and each process.
Then communication of boundary data between processes is done and the process repeats.


\end{document}